\documentclass[conference]{IEEEtran}
\usepackage{enumitem}
\usepackage{cite}

\title{PCA-Based Image Compression}

\author{
\IEEEauthorblockN{Owen Sowatzke}
\IEEEauthorblockA{\textit{Electrical Engineering Department} \\
\textit{University of Arizona}\\
Tucson, USA \\
osowatzke@arizona.edu}
\and
\IEEEauthorblockN{Scott Thoesen}
\IEEEauthorblockA{\textit{Electrical Engineering Department} \\
\textit{University of Arizona}\\
Tucson, USA \\
thoesens@arizona.edu}}

\begin{document}
	\maketitle
	
	\section{Introduction}
	This project aims to explore principle component analysis (PCA), its relationship to singular value decomposition (SVD), and its application to the field of image compression [ref]. The topic of digital image compression has become more important than ever with the advent of smartphones and social media, with millions of images being taken, stored, transferred, and copied daily. Reducing the digital footprint of data is therefore necessary to both satisfy end-user demands and decrease the demand on limited computing and storage resources. Although image compression is interesting and generally important, it is not of particular importance to the authors. What is important to the authors is the more generalized use of PCA as a tool for data analysis and dimensionality reduction. Image compression was selected for this project to visually demonstrate the connection that is made by PCA between real-world data and linear algebra.

	\section{Background}
	How does the selected material for your project relate to central themes from
ECE 501b? Are there particular techniques or is there theory from ECE 501b that is used in the
selected material? What topics from ECE 501b helped you better understand the material and
ideas in your project? Include references of all sources in the Reference section at the end of your
report.

	\subsection{Theory}
	Placeholder text

	\section{Results}
	(Data, Analysis, Design, etc.) Using the 25,000,000,000 Eigenvector paper as a
guide, develop some exercises, questions, analysis, or examples that were inspired or motivated
by your project. For example, what assumptions were made in your project material? What if
certain assumptions are neglected or do not exist? Can you provide analysis that supplements the
analysis in the selected project material? Or, can you provide analysis that extends the analysis in
the project material? You might also develop numerical examples that help to provide insight
into the ideas discussed in the selected material or that demonstrate the application of the theory
in your project. The example you create might consist of a MATLAB simulation or MATLAB
analysis.

	\section{Conclusion}
	Provide a summary and conclusion section in the report. How well did your
example correlate or explain the concepts in the selected material? How would you modify your
example to make it better? How did your selected project enhance your understanding of ECE
501b material?
	
	\nocite{shlens_2014_tutorial}
	\nocite{jaradet_svd_image_compression}
	\nocite{omar_image_compression}
	\newpage
	\bibliography{sources}{}
	\bibliographystyle{ieeetr}
\end{document}
