\documentclass{article}
\usepackage{enumitem}

\title{Reflective Paragraph}
\author{Owen Sowatzke}
\date{December 6, 2023}
	
\begin{document}
	\maketitle

	The project helped me better understand topics from ECE 501B such as orthonormal basis vectors, inner products, eigenvalue decomposition, and singular value decomposition (SVD). Beyond the standard course content, I also learned about principle component analysis (PCA) and how the SVD can be used to implement it. There are existing course resources that discuss PCA and a lecture in which image compression with SVD was briefly discussed. As part of the project, I explained the relationship between these concepts and specifically showed how image compression with SVD was equivalent to performing PCA and discarding less important principle components. I think this relationship may be beneficial to include in the ECE 501B material, specifically in the extra lecture on PCA. Overall, I enjoyed performing research, learning from the selected sources, and implementing the image compression algorithm. One of the few things I did not enjoy was having to explicitly call out relationships between the project material and ECE 501B in the project report.
	
	Because I worked with Scott Thoesen on the project, a division of work is also included. Scott and I both worked on the project code and project report. This ensured that each of us made roughly even contributions. We did assign sections of the report and code for each project member to work on. However, we also made edits to each others code and report sections. As such, our contributions are not solely limited to our assigned sections. Regarding the code division, I created functions to perform PCA-based compression, reconstruct compressed images, and compute the compression ratio. Scott, then, created functions to compute the peak signal to noise ratio and plot the figures included in the report. Regarded the report division, I wrote the abstract, the principle component analysis section, and the image compression using PCA section. Scott, then, wrote the introduction, the background motivation section, the singular value decomposition section, the compressibility upper bounds section, the results section, and the conclusion.
\end{document}
